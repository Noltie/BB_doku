\documentclass[12pt,a4paper]{book}
\usepackage[utf8]{inputenc}
\usepackage[german]{babel}
\usepackage[T1]{fontenc}
\usepackage{amsmath}
\usepackage{amsfonts}
\usepackage{amssymb}
\usepackage[left=25mm,right=20mm,top=25mm,bottom=30mm]{geometry}
\author{Tobias Noltenhans}
\title{Dokumentation Bedienen und Beobachten WS 14/15}
\date{\today}
\begin{document}
\maketitle
\tableofcontents
\chapter{Aufgabenstellung}
es ist möglich aber nicht schlau
\chapter{Programmierung}
Hier sehen sie, wie das Ganze in LAtex Programmiert wurde
\section{Aufgabenstellung}
Hier ist die Aufgabenstellung von Onkel Langmann
\section{Stationsbeschreibung}
\section{Aufbau und Struktur der Webseiten}
\end{document}
